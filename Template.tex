\documentclass[xcolor={table}, aspectratio=169]{beamer}
%
% Choose how your presentation looks.
%
% For more themes, color themes and font themes, see:
% http://deic.uab.es/~iblanes/beamer_gallery/index_by_theme.html
%
\defbeamertemplate{footline}{my frame number}
{
    \hfill
    \usebeamercolor[fg]{page number in head/foot}
    \usebeamerfont{page number in head/foot}
    \raisebox{0.25cm}[0pt][0pt]{
        \insertframenumber\kern1em}
}
\mode<presentation>
{
    \usetheme{default}
    \usecolortheme{seagull}
    \usefonttheme{default}
    \setbeamertemplate{caption}[numbered]
    \setbeamertemplate{footline}[my frame number]
    \setbeamertemplate{navigation symbols}{}
    \addtobeamertemplate{frametitle}{\vskip+0.5ex}{}
}
% \AtBeginSection[] {
%     \begin{frame}<beamer>
%     \frametitle{Outline}
%     \tableofcontents[currentsection]
%     \end{frame}
% }
% \AtBeginSubsection[] {
%     \begin{frame}<beamer>
%     \frametitle{Outline}
%     \tableofcontents[currentsection, currentsubsection]
%     \end{frame}
% }

\usepackage[english]{babel}
\usepackage[export]{adjustbox}
\usepackage[permil]{overpic}
\usepackage{threeparttable}
\usepackage{graphicx}
\usepackage{subcaption}
\usepackage{setspace}
\usepackage{ulem}
\usepackage{amssymb}
\usepackage{subcaption}
\usepackage{psfrag}
%\usepackage{natbib}
\usepackage[autocite=superscript,backend=bibtex, style=authoryear]{biblatex}
\addbibresource{refs.bib}

\usepackage{multirow}
\usepackage{hyperref}
\usepackage{graphicx}
\usepackage{color}
% \usepackage{cleveref}
\usepackage[font=scriptsize]{caption}
\usepackage{booktabs}
\usepackage{amsmath}

\usepackage{siunitx}

\sisetup{output-exponent-marker=\ensuremath{\mathrm{e}}}



%\usepackage{fontspec}

\graphicspath{{stevens/}{figures/}}

\makeatletter
\def\beamer@framenotesbegin{% at beginning of slide
     \usebeamercolor[fg]{normal text}
      \gdef\beamer@noteitems{}%
      \gdef\beamer@notes{}%
}
\makeatother

\usebackgroundtemplate
{
    \begin{tabular}{@{}c@{}}
        \includegraphics[width=\paperwidth]{presentation_top.png} \\
        \rule{0pt}{0.74\paperheight} \\
        \includegraphics[width=\paperwidth]{presentation_bottom.png}
    \end{tabular}
}

\vspace{1cm}

\title[Lit Review]
      {Crypto Market Analysis, ChainSure+, FOMC Sentiment Analysis and Rate Predictions}

\author[Author] % (optional, use only with lots of authors)
{Andre Sealy}
% - Use the \inst{?} command only if the authors have different
%   affiliation.

\institute[Stevens Institute] % (optional, but mostly needed)
{
  Financial Engineering 800\\
  Stevens Institute of Technology\\
%System description details at \cite{w2020shift}}
}
\date{\\
{\footnotesize December 9th, 2025}}

\subject{Talk at Venue}


%\author[Ionut Florescu]{\texorpdfstring{\footnotesize Ionu\c{t} Florescu \\
%       \vspace*{0.5\baselineskip}
%     {\footnotesize Committee: \\
%      Dr. George Calhoun, Dr. Dragoș Bozdog, \\
%      Dr. Emmanuel Hatzakis, Dr. Rupak Chatterjee}}{Ionu\c{t} Florescu}}
% \institute{\footnotesize Financial Engineering}
% \date{{\footnotesize September 12, 2020}}

\begin{document}

{
\setbeamertemplate{footline}{}
\usebackgroundtemplate
{
	\begin{tabular}{@{}c@{}}
		\begin{overpic}[width=\paperwidth]{title_top.png}
			\put(25,-35){\includegraphics[height=0.15\paperheight]{title_logo.png}}
		\end{overpic} \\
		\rule{0pt}{0.74\paperheight}                     \\
		\includegraphics[width=\paperwidth]{title_bottom.png}
	\end{tabular}
}

\begin{frame}

	\titlepage

\end{frame}
}

\begin{frame}{Overview}
	\begin{enumerate}
		\item  Crypto Currency Ecosystem and Interlinkages
		      \newline
		\item ChainSure+
		      \newline
		\item FOMC Sentiment Analysis and Rate Predictions
	\end{enumerate}
\end{frame}

\begin{frame}
	\centering
	\vfill
	{\Large \textbf{Crypto Currency Ecosystem and Interlinkages}}
	\vfill
\end{frame}

\begin{frame}{Overview of the Crypto Market}
\begin{table}[ht]
	\centering
	\caption{Market Comparisons of the Top Ten Crytocurrencies on December 30th, 2022}
	\begin{tabular}[t]{lcccc}
		\toprule
		Company Name & Market Cap & Last Price & 1D Pct Chg& 1M Pct Chg \\
		\midrule
    Bitcoin (BTC-USD)  & \$320\,\text{B} & \$16{,}600 & -0.3\% & -3\%  \\
    Ethereum (ETH-USD) & \$145\,\text{B} & \$1{,}200  & +0.5\% & -7\%  \\
    Tether (USDT-USD)   & \$66\,\text{B}  & \$1.00     &  0.0\% &  0\%  \\
    USD Coin (USDC-USD) & \$44\,\text{B}  & \$1.00     &  0.0\% &  0\%  \\
    BNB (BNB-USD)      & \$39\,\text{B}  & \$245      & -0.5\% & -11\% \\
    XRP (XRP-USD)      & \$17\,\text{B}  & \$0.34     & -1.0\% & -15\% \\
    Cardano (ADA-USD)  & \$9\,\text{B}   & \$0.24     & -0.5\% & -18\% \\
    Dogecoin (DOGE-USD) & \$10\,\text{B}  & \$0.07     & -1.0\% & -18\% \\
    TRON (TRX-USD)     & \$5\,\text{B}   & \$0.055    & -1.0\% & -12\% \\
		\bottomrule
	\end{tabular}\label{tab:market_table}
\end{table}
\end{frame}

\begin{frame}{Overview of the Crypto Market}
We find the ETH, ADA, XRP, BNB, and DOGE are the most correlated with BTC, so we use these crypto-assets.
\begin{figure}[H]
   \centering
  \includegraphics[width=0.8\linewidth]{pics/correlation_all_crypto.png}
\caption{Correlation Matrix of the asset prices of Cryptocurrencies}
\label{fig:correlation_all_crypto}
\end{figure}
\end{frame}

\begin{frame}{Overview of the Crypto Market}
\begin{figure}[H]
    \centering
    \begin{subfigure}[b]{0.48\textwidth}
        \centering
        \includegraphics[width=1\textwidth]{pics/crypto_assets.png}
        \caption{One-year trend of crypto-assets}
        \label{fig:asset_trend}
    \end{subfigure}
    \hfill
    \begin{subfigure}[b]{0.48\textwidth}
        \centering
        \includegraphics[width=1\textwidth]{pics/crypto_assets_returns.png}
        \caption{Cumulative returns of crypto-assets}
        \label{fig:asset_return}
    \end{subfigure}
    \caption{Trends of crypto asset prices and cumulative returns}
    \label{fig:trend_and_returns}
\end{figure}
\end{frame}

\begin{frame}{Overview of the Crypo Market}
  \begin{figure}
     \centering
    \includegraphics[width=0.45\linewidth]{pics/correlation_crypto.png}
  \caption{Correlation Matrix of the daily returns of the crypto market}
  \label{fig:correlation_crypto}
  \end{figure}
\end{frame}

\begin{frame}{Overview of the Crypto Market}
Overall Trend 
\begin{itemize}
  \item  We find the most crypto coins are highly correlated with each other, with the exception of stable coins and niche/utility coins.
  \newline
  \item Similar to equities, most crypto coins are highly correlated with one another. Trends in the largest coins tend to lead to trends in all coins.
  \newline
\end{itemize}
\end{frame}

\begin{frame}
	\centering
	\vfill
	{\Large \textbf{Descriptive Statistics of the Crypto Market}}
	\vfill
\end{frame}

\begin{frame}{Descriptive Statistics of the Crypto Market}
  We use the following descriptive statistics to analyze the crypto market.
  
  \begin{align}
    &\text{Sample Mean}:\;\hat{\mu}_x=\frac{1}{T}\sum_{t=1}^{T}x_t,\\
    &\text{Sample Standard Deviation}:\;\hat{\sigma}_x=\sqrt{\frac{1}{T-1}\sum_{t=1}^{T}\left(x_t-\hat{\mu}_2\right)^2},\\
    &\text{Sample Skewness}:\;\hat{S}(x)=\frac{1}{\left(T-1\right)\hat{\sigma}^3_x}\sum_{t=1}^{T}\left(x_t-\hat{\mu}_x\right)^3,\\
    &\text{Sample Kurtosis}:\;\hat{K}(x)=\frac{1}{\left(T-1\right)\hat{\sigma}^4_x}\sum_{t=1}^{T}\left(x_t-\hat{\sigma}_x\right)^4
  \end{align}
\end{frame}

\begin{frame}{Descriptive Statistics of the Crypto Market}
\begin{table}[ht]
    \captionsetup{skip=2pt} % space between caption and table (try 0–4pt)
    \centering
    \caption{Descriptive statistics of cryptocurrencies (\textit{Daily Simple Returns \%})}
    \label{tab:descriptive}
    \resizebox{\textwidth}{!}{%
        \begin{tabular}{lcccccc}
            \toprule
            Security & Mean ($\hat{\mu}_x$) & Std Dev ($\hat{\sigma}_x$) &
            Skewness ($\hat{S}(x)$) & Kurtosis ($\hat{K}(x)$) & Min & Max \\
            \midrule
            BTC  & -0.02 & 3.44 & -0.2660 &  5.96 & -15.97 & 14.54 \\
            ETH  & -0.02 & 4.67 & -0.0491 &  7.95 & -17.45 & 18.11 \\
            ADA  & -0.03 & 4.92 & 0.1279  &  9.17 & -18.46 & 18.48 \\
            XRP  & -0.01 & 4.60 & 0.3946  &  9.53 & -19.51 & 22.24 \\
            BNB  &  0.00 & 3.87 & -0.5519 &  8.66 & -18.56 & 13.95 \\
            DOGE &  0.00 & 5.77 & 1.3236  & 12.39 & -22.02 & 44.94 \\
            \bottomrule
        \end{tabular}%
    }
\end{table}
\end{frame}

\begin{frame}{Descriptive Statistics of the Crypto Market}
Overall Descriptive Analysis 
\begin{itemize}
  \item Cryptocurrencies all show high levels of standard deviation, which is inline with what we expect from the overall risk of the asset. 
  \newline
  \item Cryptocurrencies all exhitbit a high level of kurtosis, indicating that high levels of single day gains/losses happen at greater frequencies. 
  \newline
  \item We find that BTC has the smallest amount of risk and DOGE has the greatest amount of risk.
\end{itemize}
\end{frame}

\begin{frame}
	\centering
	\vfill
	{\Large \textbf{Crypto in Comparison to the Overall Market}}
	\vfill
\end{frame}

\begin{frame}{Crypto and the Overall Market}
\begin{figure}
    \centering
        \includegraphics[width=0.80\linewidth]{pics/correlation_matrix_sp.png}
        \caption{Correlation of the Crypto Market and the S\&P 500}
        \label{fig:}
    \end{figure}
\end{frame}

\begin{frame}{Crypto and the Overall Market}
\begin{figure}[H]
    \centering
    \begin{subfigure}[b]{0.48\textwidth}
        \centering
        \includegraphics[width=1\textwidth]{pics/crypto_assets_sp.png}
        \caption{One-year trend of crypto-assets}
        \label{fig:asset_trend_sp}
    \end{subfigure}
    \hfill
    \begin{subfigure}[b]{0.48\textwidth}
        \centering
        \includegraphics[width=1\textwidth]{pics/crypto_assets_returns_sp.png}
        \caption{Cumulative returns of crypto-assets}
        \label{fig:asset_return_sp}
    \end{subfigure}
    \caption{Trends of crypto asset prices and cumulative returns}
    \label{fig:trend_and_returns_sp}
\end{figure}
\end{frame}

\begin{frame}{Crypto and the Overall Market}
\begin{figure}[H]
   \centering
  \includegraphics[width=.5\linewidth]{pics/correlation_crypto_sp.png}
  \caption{Correlation Matrix for selected cryptocurrencies with S\&P 500 }
\label{fig:correlation_crypto_new_sp}
\end{figure}
\end{frame}

\begin{frame}{Crypto and the Overall Market}
\begin{figure}[H]
   \centering
  \includegraphics[width=0.70\linewidth]{pics/daily_returns_sp.png}
  \caption{Daily Returns of cryptocurrencies and the S\&P 500}
\label{fig:daily_returns_sp}
\end{figure}
\end{frame}


\begin{frame}{Descriptive Statistics of the Crypto Market}
Crypto in relation with the overall market 
\begin{itemize}
  \item As expected, the broader market is more stable than the overall crypto market; however, periods of volatility tend to differ between the two asset classes.
  \newline
  \item Cryptocurrencies like DOGE and XRP tend to be more volatile than other coins
  \newline
  \item All cryptocurrencies exhibit the same volatility clustering around the periods, especially around November 11th, 2022, which is the height of the FTX Collapse.
\end{itemize}
\end{frame}

\begin{frame}
	\centering
	\vfill
	{\Large \textbf{FTX Collapse}}
	\vfill
\end{frame}


\begin{frame}{The FTX Collapse}
{\normalsize

% increase the space between columns (local to this group)
\setlength{\columnsep}{20pt}  % try 10–15pt

\begin{columns}[t,totalwidth=\textwidth]
  \column{0.73\textwidth} % was 0.73
    \begin{itemize}\itemsep8pt
      \item FTX and affiliates filed for bankruptcy, confirming the exchange had a multi-billion dollar gap between assets and liabilities, and that many customers' funds were missing or illiquid.
      \item The announcement intensified volatility and downside pressure on major coins like BTC and ETH.
      \item Financial commentators widely described the event as a "Lehman moment" for the crypto market, despite the spillover into other asset classes remaining limited.
    \end{itemize}

  \column{0.4\textwidth} % was 0.4
    \vspace{-0.5cm}
    \begin{figure}
      \centering
      \includegraphics[width=\linewidth]{pics/press_release.png}
      \caption{\footnotesize FTX announcing Chapter 11 bankruptcy}
    \end{figure}
\end{columns}

}
\end{frame}

\begin{frame}{ChainSure+}
  Withdrawl freeze, a deal falls through: Nov 8 to 100
{\footnotesize

% increase the space between columns (local to this group)
\setlength{\columnsep}{20pt}  % try 10–15pt

\begin{columns}[t,totalwidth=0.9\textwidth]
  \column{0.60\textwidth} % was 0.73
    \begin{itemize}\itemsep2pt
  \item CEO Sam Bankman-Fried and Binance CEO Chengpeng Zhao stuck a deal for Binance to aquire the non-U.S. branch of FTX, including a promise to bail out FTX to prevent larger market crashes.
  \item On Nov. 8, FTX halted all non-fiat customer withdrawls. On Twitter, Bankman-Fried posted a string of apologies explaining FTX's liquidity issues and promising more transparency.
  \item On Nov. 9, walked back his decision to bailout FTX, citing mishandling customer funds, agency investigations and other issues that contributed to his decision.
  \item Bankman-Fried appeared to reference Zhao's influence on FTX's fall in a cryptic post on Twitter where he said, "Well played; you won."
    \end{itemize}

  \column{0.5\textwidth} % was 0.4
    \vspace{0.5cm}
    \begin{figure}
      \centering
      \includegraphics[width=0.9\linewidth]{pics/tweet.png}
      \caption{\footnotesize SBF's response to CZ's news}
    \end{figure}
\end{columns}
}
\end{frame}

\begin{frame}{The FTX Collapse}
  Bankruptcy and hack: Nov. 11
  \vspace{0.3cm}
\begin{itemize}
  \item On Nov. 11, FTX announced that it had filed for voluntary Chapter 11 bankruptcy proceedings for FTX, FTX.US and Alameda.
  \newline
  \item FTX.US also temporarily froze withdrawls on Nov. 11, following the bankruptcy announcement, despite reassurances that FTX.US was not affected by FTX's liquidity troubles
  \newline 
  \item FTX and FTX.US wallets were emptied on the evening of Nov. 11 in an apparent hack.
\end{itemize}
\end{frame}

\begin{frame}
	\centering
	\vfill
	{\Large \textbf{FTX Collapse}}
	\vfill
\end{frame}

\begin{frame}{ChainSure+: Decentralized Travel Insurance}
  \begin{itemize}
    \item Builds on Etherisc's Flight Delay Insurance concept.
    \item Moves from single-flight, simple payouts to a full travel-insurance product.
    \item Decentralized protocol on Ethereum using smart contracts and real-time oracles.
    \item Focus: fast, transparent, and predictable protection against flight disruptions.
    \item Designed for trips with multiple connections, international travel, and weather uncertainty.
  \end{itemize}
\end{frame}

\begin{frame}{Target Consumers \& Existing Competitors}
  \begin{itemize}
    \item Travelers in regions where traditional insurers are slow, inconsistent, or opaque.
    \item Users who value automated payouts without manual claim review.
    \item DeFi participants acting as liquidity providers, earning returns from premiums.
    \item Closest decentralized competitor: Etherisc Flight Delay Insurance (single-flight focus).
    \item Centralized competitors: traditional insurers (e.g., Allianz, AIG) and airline compensation schemes.
    \item ChainSure+ aims to be a more complete, consistent, and user-friendly alternative.
  \end{itemize}
\end{frame}

\begin{frame}{Benefits to Travelers}
  \begin{itemize}
    \item Single policy can cover an entire multi-leg trip (no need for multiple policies).
    \item Includes coverage for weather-related cancellations via real-time weather data.
    \item Tiered payout structure: compensation scales with length / severity of delay.
    \item Dynamic, risk-based premiums using historical airline and airport reliability data.
    \item Fully automated payouts triggered by oracle data; no forms, documents, or manual approval.
  \end{itemize}
\end{frame}

\begin{frame}{ChainSure+ vs.\ Centralized Insurance}
  \begin{itemize}
    \item Smart contracts automate payouts and remove manual claims processing.
    \item On-chain policy rules and payout conditions are transparent and immutable.
    \item Oracle-based data feeds reduce fraud and opportunistic claims.
    \item Lower administrative overhead $\Rightarrow$ potential for more affordable premiums.
    \item Policies implemented as NFTs:
      \begin{itemize}
        \item Transferable if travel plans change.
        \item Resale or reassignment possible in secondary markets.
      \end{itemize}
  \end{itemize}
\end{frame}

\begin{frame}{Smart-Contract Design on Ethereum}
  \begin{itemize}
    \item \textbf{Policy Contract (NFT):}
      \begin{itemize}
        \item Stores flight details, coverage terms, and ownership.
        \item Enables transfer of policies between users.
      \end{itemize}
    \item \textbf{Risk Pool Contract:} Holds liquidity, receives premiums, pays claims.
    \item \textbf{Payout Logic Contract:} Maps delay / cancellation data to tiered payouts.
    \item \textbf{Pricing Contract:} Computes dynamic premiums from historical reliability data.
    \item \textbf{Oracles (e.g., Chainlink):}
      \begin{itemize}
        \item Provide flight status, weather, and performance metrics.
        \item Trigger automatic payouts when conditions are met.
      \end{itemize}
  \end{itemize}
\end{frame}

\begin{frame}{Economic Viability \& Deployment Choices}
  \begin{itemize}
    \item Economically viable on Ethereum, especially \textbf{Layer 2} networks (lower gas fees).
    \item Oracle calls + payouts remain cost-effective on rollups; supports frequent, small policies.
    \item Flight-disruption risks follow relatively stable statistical patterns:
      \begin{itemize}
        \item Supports predictable pool management and premium setting.
      \end{itemize}
    \item Liquidity providers earn yield from premiums in exchange for bearing risk.
    \item Extensible to other EVM-compatible chains (Polygon, Arbitrum, Optimism, Avalanche).
    \item Less suitable / more complex on Solana (new language, oracle support) and not viable on Bitcoin.
    \item Conclusion: Ethereum L2s are the most practical, cost-efficient home for ChainSure+.
  \end{itemize}
\end{frame}


\begin{frame}
	\centering
	\vfill
	{\Large \textbf{Analyzing the Federal Open Market Committe}}
	\vfill
\end{frame}

\end{document}